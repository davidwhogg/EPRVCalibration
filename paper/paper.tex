% This document is part of the EPRVCalibration project
% Copyright 2019 the authors. All rights reserved.

\documentclass[12pt, letterpaper]{article}

% typesetting words
\newcommand{\project}[1]{\textsl{#1}}
\newcommand{\acronym}[1]{{\small{#1}}}
\newcommand{\expres}{\project{\acronym{EXPRES}}}

% margins and page setup, etc
\addtolength{\textheight}{1.00in}
\addtolength{\topmargin}{-0.50in}
\sloppy\sloppypar\raggedbottom\frenchspacing

\begin{document}

\section*{Method}

The idea is that the wavelength solution is going to live in a
low-dimensional space, where the degrees of freedom are set by the
degrees of freedom of the spectrograph hardware.

First, some definitions.
The way I (Hogg) think about this is the following:
Given an exposure $n$, and order $m$, there is a relationship between
the two-dimensional $(x,y)$-position on the detector and the
wavelength $\lambda$
\begin{equation}
\lambda = f(x,y;m,\theta_{n})
\quad ,
\end{equation}
where $\theta_{n}$ is a big blob of parameters for this exposure.
This is not precisely how the \expres\ team thinks about the problem,
I (Hogg) think.
For now I am going to ignore that fact, but I think what I say here
could be adapted to any \expres\ conventions.
Right now in the \expres\ pipeline the $\theta_{n}$ comprises the
amplitudes of some 9th-ish-order polynomial in $x$ and $n$, possibly
ignoring $y$?.

If the space of all calibration possibilities is in fact
$K$-dimensional (where I am thinking of this as being a small integer,
like $K=3$ or $K=8$ or thereabouts), and if the calibration variations are so
small that we can linearize, then the function $f(x,y;m,\theta_{n})$ could
be replaced with a tiny model
\begin{equation}
\lambda = g_0(x,y;m) + \sum_{k=1}^K a_{nk}\,g_k(x,y;m)
\quad ,
\end{equation}
where
$g_0(x,y;m)$ is the fiducial or mean or standard calibration of the
spectrograph,
the $a_{nk}$ are scalar amplitudes,
and the $g_k(x,y;m)$ are basis functions expressing the ``directions''
in calibration space that the spectrograph can depart from the
fiducial calibration.
The challenge is to learn these basis functions from the data, and get
the $K$ amplitudes $a_{nk}$ for every exposure $n$.

\end{document}
